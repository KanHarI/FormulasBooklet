\documentclass{article}
\usepackage[landscape]{geometry}
\usepackage{url}
\usepackage{multicol}
\usepackage{amsmath}
\usepackage{esint}
\usepackage{amsfonts}
\usepackage{tikz}
\usetikzlibrary{decorations.pathmorphing}
\usepackage{amsmath,amssymb}

\usepackage{colortbl}
\usepackage{xcolor}
\usepackage{mathtools}
\usepackage{amsmath,amssymb}
\usepackage{enumitem}
\makeatletter

\newcommand*\bigcdot{\mathpalette\bigcdot@{.5}}
\newcommand*\bigcdot@[2]{\mathbin{\vcenter{\hbox{\scalebox{#2}{$\m@th#1\bullet$}}}}}
\makeatother

\title{STAT 251 Formula Sheet}
\usepackage[brazilian]{babel}
\usepackage[utf8]{inputenc}

\advance\topmargin-.8in
\advance\textheight3in
\advance\textwidth3in
\advance\oddsidemargin-1.45in
\advance\evensidemargin-1.45in
\parindent0pt
\parskip2pt
\newcommand{\hr}{\centerline{\rule{3.5in}{1pt}}}
%\colorbox[HTML]{e4e4e4}{\makebox[\textwidth-2\fboxsep][l]{texto}



\DeclareMathOperator{\tr}{tr}



\begin{document}

\begin{center}{\huge{\textbf{Useful formulas}}}\\
\end{center}
\begin{multicols*}{2}

\tikzstyle{mybox} = [draw=black, fill=white, very thick,
    rectangle, rounded corners, inner sep=10pt, inner ysep=10pt]
\tikzstyle{fancytitle} =[fill=black, text=white, font=\bfseries]


%------------ Basic definitions
\begin{tikzpicture}
\node [mybox] (box){%
    \begin{minipage}{0.46\textwidth}
    
        $$ 0! = 1 \qquad \forall n \in \mathbb{N}^+, n! = n(n-1)! = n\cdot(n-1)\cdots2\cdot 1 $$
        $$ (a)_n = \prod_{k=a}^{a+n-1} k = a\cdot(a+1)\cdots(a+n-1), \qquad (a)_0 = 1, \qquad (1)_n = n! $$
        $$ \forall n > k, {n \choose k} = \frac{n!}{n!(n-k)!} = \frac{(n-k+1)_k}{(1)_k} = \frac{(n-k+1)(n-k+2)\cdots n}{1\cdot2\cdots k} $$
        $$ e^x = \sum_{n=0}^\infty \frac{x^n}{n!} = \lim_{n\rightarrow\infty} \left(1+\frac{x}{n}\right)^n, \qquad e \approx 2.718281828 $$
        $$ i^2 = -1 $$
    	$$ e^{ix} = \cos x + i \sin x $$
    	$$ \sin x = \frac{e^{ix} - e^{-ix}}{2i}, \qquad \cos x = \frac{e^{ix}+e^{-ix}}{2} $$
    	$$ \tan x = \frac{\sin x}{\cos x} $$
    	$$ \pi = 4 \left( \sum_{n=0}^\infty \frac{(-1)^n}{2n+1} \right) \approx 3.141592654 $$
%     	$$ \text{Bernoulli numbers:} \qquad B_0^+ = B_0^{-} = B_0 = 1 $$
%     	$$ B_n^+ = 1 - \sum_{k=0}^{n-1} {n \choose k} \frac{B_k^+}{n - k -1} $$
%     	$$ B_n^{-} = (-1)^nB_n^{+} $$
%     	$$ \forall n \in \mathbb{N}, B_{2n} = B_{2n}^+ = B_{2n}^- $$
	
    \end{minipage}
};
%------------ Basic definitions Header ---------------------
\node[fancytitle, right=10pt] at (box.north west) {Basic definitions};
\end{tikzpicture}

%------------ Binomial formulas
\begin{tikzpicture}
\node [mybox] (box){%
    \begin{minipage}{0.46\textwidth}
    
        $$ (x\pm y)^2 = x^2 \pm 2xy + y^2, \qquad (x\pm y)^3 = x^3 \pm 3x^2y + 3xy^2 \pm y^3 $$
        $$ (x\pm y)^n = \sum_{k=0}^n {n \choose k} (\pm 1)^k x^{n-k}y^k $$
        $$ a_1+a_2+\dots+a_m = n \implies {n \choose a_1, a_2, \dots, a_m} = \frac{n!}{a_1!a_2!\cdots a_m!} $$
        $$ (x + y + z)^n = \sum_{i, j, k \in \mathbb{N}^3 | i + j + k = n} {n \choose i,j,k}x^i y^j z^k $$
        $$ (x_1 + x_2 + \dots + x_m)^n = \sum_{a_1, a_2, \dots, a_m \in \mathbb{N}^m | a_1 + a_2 + \dots + a_m = n} {n \choose a_1, a_2, \dots a_m} x_1^{a_1} x_2^{a_2} \cdots x_m^{a_m} $$
        $$ \forall y \in \mathbb{R}, x \in (-1, 1) \implies (1+x)^y = \sum_{k=0}^\infty {y \choose k}x^k = \sum_{k=0}^\infty \frac{(y - k + 1)_k}{k!} x^k $$
        $$ {n \choose k} = {n \choose n-k}, \qquad \sum_{k=0}^n {n \choose k} = 2^n, \qquad {n+1 \choose k+1} = {n \choose k} + {n \choose k + 1} $$
	
    \end{minipage}
};
\node[fancytitle, right=10pt] at (box.north west) {Binomial formulas};
\end{tikzpicture}

%------------ Trig identities
\begin{tikzpicture}
\node [mybox] (box){%
    \begin{minipage}{0.46\textwidth}
    
        $$ \sin^2 x + \cos^2 x = 1 $$
        $$ \sin (-x) = -\sin x, \qquad \cos(-x) = \cos x, \qquad \tan(-x) = -\tan x $$
        $$ \sin \left(\frac{\pi}{2} \pm x\right) = \cos x, \qquad \sin(\pi \pm x) = \mp \sin x $$
        $$ \cos \left(\frac{\pi}{2} \pm x\right) = \mp \sin x, \qquad \cos(\pi \pm x) = -\cos x $$
        $$ \sin x = \sum_{n=0}^{\infty} \frac{(-1)^n x^{2n+1}}{(2n+1)!}, \qquad \cos x = \sum_{n=0}^{\infty} \frac{(-1)^n x^{2n}}{(2n)!} $$ 
        $$ \forall x \in \left(-\frac{\pi}{2}, \frac{\pi}{2}\right), \tan x = \sum_{n=1}^\infty \frac{B_{2n}(-4)^n(1-4^n)x^{2n-1}}{(2n)!} = x + \frac{x^3}{3} + \frac{2x^5}{15} + \frac{17 x^7}{315} + O(x^9) $$
    	$$ \sin 2x = 2\sin x \cos x $$
    	$$ \cos 2x = \cos^2 x - \sin^2 x = 2 \cos^2 x - 1 = 1 - 2\sin^2 x $$
    	$$ \tan 2x = \frac{2\tan x}{1 - \tan^2 x} $$
    	$$ \sin (nx) = \sum_{k\text{ odd}} (-1)^\frac{k-1}{2} {n \choose k}\cos^{n-k} x \sin^k x $$
    	$$ \cos(nx) = \sum_{k\text{ even}} (-1)^\frac{k}{2} {n \choose k}\cos^{n-k} x \sin^k x $$
    	$$ \sin(x \pm y) = \sin x \cos y \pm \sin y \cos x $$
    	$$ \cos(x \pm y) = \cos x \cos y \mp \sin x \sin y $$
    	$$ \tan(x \pm y) = \frac{\tan x \pm \tan y}{1 \mp \tan x \tan y} $$
    	$$ \tan \left(\frac{\pi}{4} \pm x\right) = \frac{\frac{1}{\tan x} \mp 1}{1 \pm \frac{1}{\tan x}} $$
    	$$ \sin \frac{x}{2} = \pm \sqrt{\frac{1-\cos x}{2}}, \qquad \cos \frac{x}{2} = \pm \sqrt{\frac{1+\cos x}{2}} $$
    	$$ \tan \frac{x}{2} = \frac{\sin x}{1 + \cos x} $$
    	$$ \tan \left(\frac{x \pm y}{2}\right) = \frac{\sin x \pm \sin y}{\cos x + \cos y} $$
    	$$ \sin x \pm \sin y = 2\sin \left( \frac{x\pm y}{2} \right) \cos\left(\frac{x \mp y}{2}\right) $$
	
    \end{minipage}
};
\node[fancytitle, right=10pt] at (box.north west) {Trig identities};
\end{tikzpicture}

%------------ Matrices
\begin{tikzpicture}
\node [mybox] (box){%
    \begin{minipage}{0.46\textwidth}
        If the eigenvalues of a a matrix $M$ are $\lambda_1, \lambda_2, \dots, \lambda_n$ than:
        $$ |M| = \prod_{m=1}^n \lambda_m, \qquad \tr M = \sum_{m=1}^n \lambda_m $$
        Every square matrix over the complex plane can be expressed in jordan cannonical form:
        $$ J_{\lambda,1} = \begin{bmatrix}\lambda \end{bmatrix}, \qquad J_{\lambda, 2} = \begin{bmatrix}\lambda & 1 \\ 0 & \lambda \end{bmatrix}, \qquad J_{\lambda, n} = \begin{bmatrix} \lambda & 1 & 0 & \cdots & 0 & 0 \\ 0 & \lambda & 1 & \cdots & 0 & 0 \\ 0 & 0 & \lambda & \cdots & 0 & 0 \\ \vdots & \vdots & \vdots & \ddots & \vdots & \vdots \\ 0 & 0 & 0 & \cdots & \lambda & 1 \\ 0 & 0 & 0 & \cdots & 0 & \lambda \end{bmatrix} $$
        $$ \forall M \in \mathbb{M}_{n \times n}(\mathbb{C}), \exists P \in \mathbb{M}_{n \times n}(\mathbb{C}), M = P^{-1} \begin{bmatrix} J_{\lambda_1, n_1} & & & \\ & J_{\lambda_2, n_2} & & \\ & & \ddots & \\ & & & J_{\lambda_m, n_m} \end{bmatrix} P $$
        $$ e^M = \sum_{n=0}^\infty \frac{M^n}{n!} $$
        $$ \left|e^M\right| = e^{\tr M} $$
    	A random matrix over the complex field is diagonalizable with probability 1. If $M$ is diagonalizable: \\
    	$$ M^n = \left(P^{-1}\begin{bmatrix}\lambda_1 & 0 & \cdots & 0  \\ 0 & \lambda_2 & \dots & 0 \\ \vdots & \vdots & \ddots & \vdots \\ 0 & 0 & \cdots && \lambda_m \end{bmatrix}P\right)^n = P^{-1} \begin{bmatrix}\lambda_1^n & 0 & \cdots & 0  \\ 0 & \lambda_2^n & \dots & 0 \\ \vdots & \vdots & \ddots & \vdots \\ 0 & 0 & \cdots & \lambda_m^n \end{bmatrix} P $$
	
    \end{minipage}
};
\node[fancytitle, right=10pt] at (box.north west) {Matrices};
\end{tikzpicture}

%------------ Formal power series
\begin{tikzpicture}
\node [mybox] (box){%
    \begin{minipage}{0.46\textwidth}

    	$$ \left(\sum_{n=0}^\infty a_n x^n\right) \left(\sum_{n=0}^\infty b_n x^n\right) = \sum_{n=0}^\infty \left(\left(\sum_{l=0}^n a_l b_{n-l} \right)x^n\right) $$
    	$$ \left(\sum_{n \in \mathbb{Z}} a_n x^n\right)\left(\sum_{n \in \mathbb{Z}} b_n x^n\right) = \sum_{n,m \in \mathbb{Z}^2} a_n b_m x^{n+m} $$
	
    \end{minipage}
};
\node[fancytitle, right=10pt] at (box.north west) {Formal power series};
\end{tikzpicture}

%------------ Basic definitions
\begin{tikzpicture}
\node [mybox] (box){%
    \begin{minipage}{0.46\textwidth}

    	$$ \text{Riemann's zeta:} \quad \forall z \in \mathbb{C}, \Re z > 1, \quad \zeta(z) = \sum_{n=1}^\infty \frac{1}{n^z} $$
    	$$ \text{Dirichlet's eta:} \forall z \in \mathbb{C}, \Re z > 0, \quad \eta(z) = \sum_{n=1}^\infty \frac{(-1)^{n+1}}{n^z} = \left(1 - 2^{1-z} \right)\zeta(z) $$
    	Lambert's W - the inverse of $ x\cdot e^x $, the bigger one if there are 2 inverses:
    	$$ \forall x > -\frac{1}{e}, \quad W_0(x e^x) = x $$
    	The smaller branch of Lambert's W:
    	$$ \forall x \in \left[-\frac{1}{e}, -1\right], \quad W_{-1}(xe^x) = x $$
    	$$ \forall n \in \mathbb{N}, n! = \Gamma(n+1) = \Gamma(z) = \int_0^\infty t^{z-1}e^{-t} dt $$
	
    \end{minipage}
};
\node[fancytitle, right=10pt] at (box.north west) {Special functions};
\end{tikzpicture}

\end{multicols*}
\end{document}
